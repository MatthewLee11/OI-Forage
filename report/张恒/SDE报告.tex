\documentclass{article}
\usepackage[UTF8]{ctex}
\usepackage{graphicx}
\usepackage{framed}
\usepackage{color}
\usepackage{tabulary}
\usepackage{amsmath}
\usepackage{listings}
\usepackage{wrapfig}

\begin{titlepage}
    \title{随机微分方程}
    \author{张恒  21721261}
\end{titlepage}
\begin{document}
\maketitle
\newpage
\tableofcontents
\newpage


\section{基本概念}
\subsection{随机过程}
随机过程的定义如下:

\textbf{随机过程定义:}设 $(\Omega ,{\mathcal  {F}},P)$为一概率空间,另设集合T为一指标集合,如果对于所有$t\in T$,均有一随机变量 $X(t,\omega)$定义于概率空间$(\Omega ,{\mathcal  {F}},P)$,则集合$\{X(t,\omega)|t\in T\}$为一随机过程。

其中,对于固定的$\omega$, 比如$\overline{\omega}$, $\{X(t,\overline{\omega}), t \geq 0\}$被称为\textbf{路径(path)}或\textbf{轨迹(trajectory)};对于固定的t,比如$\overline{t}$,集合$\{X(\overline{t},\omega), \omega \in \Omega\}$是时刻$\overline{t}$在该随机过程中的状态集,$X(\overline{t},\omega)$也就是时刻$\overline{t}$的随机变量。

对于这个定义,我们可以通过下例来帮助理解:

\textbf{例:}我们在某一时段对某一地区成人的身高和体重(X, Y)进行随机抽样,可得出 
$$Z = (X,Y) \sim N(\mu_1,\mu_2,\rho,\sigma^2_1,\sigma^2_1)$$
$$Z(\omega) = (X(\omega),Y(\omega))$$

 这是某时段所考查随机变量的概率分布,如果\textit{每隔10年}在同一地区做同样的随机抽样,得: \\
$$Z(\omega,0)=(X(\omega,0),Y(\omega,0)) \sim N(\mu_{01},\mu_{02},\rho_0,\sigma^2_{01},\sigma^2_{02})$$

 第1个10年:$Z(\omega,1)=(X(\omega,1),Y(\omega,1)) \sim N(\mu_{11},\mu_{12},\rho_1,\sigma^2_{11},\sigma^2_{12})$

 第2个10年:$Z(\omega,2)=(X(\omega,2),Y(\omega,2)) \sim N(\mu_{21},\mu_{22},\rho_2,\sigma^2_{21},\sigma^2_{22})$

 ......

 第t个10年:$Z(\omega,t)=(X(\omega,t),Y(\omega,t)) \sim N(\mu_{t1},\mu_{t2},\rho_t,\sigma^2_{t1},\sigma^2_{t2})$ $t = 0,1,2...$

因此$\{Z(\omega,t)=(X(t),Y(t)): t \geq 0\}$表示的就是身高和体重这两个随机变量在不同时段的情况,而$Z(\omega,t)$就是一个随机过程。

\subsection{布朗运动}
布朗运动又被称为维纳过程,原指苏格兰生物学家R. Brown与1827年在显微镜下发现的花粉颗粒的不规则运动。以前都认为布朗运动的数学定义是爱因斯坦首先与1905年提出的,但其实早在1900年Bachelier就在他的博士论文中提出了布朗运动,并把布朗运动运载股票价格的研究。标准布朗运动的数学性质有以下几点:

\textbf{标准布朗运动的数学性质}设连续时间随机过程$W_t: 0 \leq t < T$ 是 $[0,T)$上的标准布朗运动,

\begin{itemize}

 \item $W_0 = 0$

 \item \textbf{独立增量性:} 对于有限个时刻$0 \leq t_1 < t_2 < ... < t_n < T$,随机变量$$W_{t_2}-W_{t_1},W_{t_3}-W_{t_2},...,W_{t_n}-W_{t_{n-1}}$$是独立的

 \item \textbf{正态性:} 对任意的$0 \leq s < t < T$, $W_t-W_s$服从均值为0,方差为t-s的正态分布

\end{itemize}


\subsection{随机微分方程}
这一节,我们将介绍一类简单的随机微分方程。随机积分在应用中往往是在 Stratonovich 微积分意义的。此微积分的设计其基本的规则与标准微积分中的相同,如链规则和分部积分法。虽然操作的规则是相同的结论仍是非常不同的。Stratonovich 随机积分可以约化到 Itō 积分, 数学上的标准的随机微分方程理论可以用于Stratonovich 随机微分方程。此外, Stratonovich 积分更适合的随机微积分在流形上的推广。很少随机积分可以解析形式解决的, 随机数值积分因此是随机积分应用的重要问题。各种数值逼近收敛到 Stratonovich 积分,使其在随机微分方程中有重要地位。但是,最广泛使用的 Langevin
方程数值解的 Eule 格式却要求使用最广泛要求 Itō 形式的方程。

考虑一般的随机微分方程$$\frac{\mathrm{d}X(t)}{\mathrm{d}t}=h[X(t),t]+g[X(t),t]R(t)$$
其中$$\overline{R(t)}=0, \overline{R(t)R(t')}=2D\delta(t-t')$$
噪声$R(t)$的每个$\delta$函数跳跃引起$U(t)$的跳跃。这是多重噪声的非线性 Langevin 方程, 它可能是
一个非线性方程,并据说有乘性噪声(multiplicative noise),因涨落力量$R(t)$是乘以未知量的函
数$R(t)$。相比之下, Langevin 方程有加性噪声(additive noise),因为涨落系数与未知量无关。
如前所述,形式上
$$R(t)=\frac{\mathrm{d}B}{\mathrm{d}t}$$
其中记号$\mathrm{d}W(t)=R(t)\mathrm{d}t$,$B(\omega,t)$是维纳过程, 增量$$B(\omega,t)=B(\omega,t+\Delta{t})-B(\omega,t)$$是齐次和正态的。

非线性 Langevin 方程必须解释为积分方程$$X(t)-X(0)=\int_0^t h[s,X(s)]\mathrm{d}s + \int_0^t g[s,X(s)]\mathrm{d}B$$第一个积分是通常意义的,第二个积分是关于随机函数关于 Wiener 过程的积分。使得随机积
分有意义的办法就是用分片常数函数近似$g[s,X(s)]$,即$$\int_0^tg[s,X(s)]\mathrm{d}B \approx \sum_{i=0}^{n-1}g_i[s,X(s)]\mathrm{d}B_i = \sum_{i=0}^{n-1}g_i[s,X(s)][B(t_{i+1})-B(t_i)], s \in [t_{i+1},t_i]$$其中${t_i}_{i=0}*n$是区间$[0,t]$的一个划分,然后考虑当区间$t_i - t_{i-1}$的最大值趋于 0 时和式的极限。
$g_i[s,X(s)], s \in [t_{i+1},t_i]$的取值有两种选择:

\begin{itemize}
\item 在区间左端点$t_i$取值
$$g_i[s,X(s)=G[t_i, X(t_i)]$$
相应的随机积分
$$\int_0^t g[s,X(s,\omega)]\mathrm{d}B(\omega,t)$$
称为Ito积分

\item 在区间端点取值求平均$$g_i[s,X(s)]=\frac{g[t_i,X(t_i)]+g[t_{i+1},X(t_{i+1})]}{2}$$
			相应的随机积分$$\int_0^t g[s,X(\omega,s)]∘\mathrm{d}B(\omega,t)$$
			称为Stratonovich积分
\end{itemize}

要注意的是,与普通积分中取分片区间任意一点积分结果仍然不变不同的是,在随机积分中,s的取值不同,最后的积分结果也会不同。



\section{随机微分方程的解}
如果随机微分方程的漂移项$\mu$和扩散项$\sigma$满足以下条件:

\begin{itemize}
	\item \textbf{全局Lipschitz条件:} 对于所有的$x,y \in R$和$t \in [0,T]$,存在常数$K<+\infty$使得$$|\mu(t,x)-\mu(t,y)|+|\sigma(t,x)-\sigma(t,y)|<K|x-y|$$

	\item \textbf{线性增长条件:} 对于所有的$x \in R$和$t \in [0,T]$,存在常数$C < +\infty$使得$$|\mu(t,x)|+|\sigma(t,x)|<C(1+|x|)$$
\end{itemize}

则随机微分方程存在唯一的、连续的强解使得:$$E\{\int_0^T|X_t|^2\mathrm{d}t\}<\infty$$

对于布朗运动$\mathrm{d}X_t=\mu\mathrm{d}t+\sigma\mathrm{d}W_t$,在给定初值$X_{t_0}$的条件下,可以求出方程的解为$$X_t=X_{t_0}+\mu(t-t_0)+\sigma(W_t-W_{t_0})$$对于几何布朗运动$\mathrm{d}X_t=\mu X_t\mathrm{d}t+\sigma X_t\mathrm{d}W_t$,在给定初值$X_{t_0}$的条件下,可以求出方程的解为$$X_t=X(t_0)exp[(\mu-\frac{1}{2}\sigma^2)(t-t_0)+\sigma(W(t)-W(t_0))]$$

不过并不是所有的随机微分方程都能解出显式解的,更多的随机微分方程只能通过迭代式求出数值解,也就是不断拟合解的路径来求数值解。通常有两种方式来对数值解进行拟合:

\begin{itemize}
	\item \textbf{Euler格式:} $$X_{i+1} = X_i + \mu (t_i,X_i)(t_{i+1-t_i}) + \sigma(t_i,X_i)(W_{i+1}-W_i)$$
	\item \textbf{Milstein格式:}	
	\begin{align}
	X_{i+1} &= X_i + \mu (t_i,X_i)(t_{i+1}-t_i)+\sigma (t_i,X_i)(W_{i+1}-W_i)\\
			&\frac{1}{2}\sigma(t_i,X_i)\sigma_y(t_i,X_i)\{(W_{i+1}-W_i)^2-(t_{i+1}-t_i)\}
	\end{align}
\end{itemize}
其中$$W_{i+1}-W_i = \sqrt{t_{i+1}-t_i}Z_{i+1}, i=0,1,...,n-1$$
	而$Z_1,...,Z_n$是相互独立的标准正态随机变量。



\section{随机微分方程的参数估计}
这一节我们介绍一种随机微分方程的参数估计方法:极大似然估计。

考虑下面的随机微分方程:$$\mathrm{d}X_t = \mu(X_t;\theta)\mathrm{d}t + \sigma(X_t;\theta)\mathrm{d}W_t$$
其中$\theta \in \Theta \subset R^p$是p维参数,$\theta_0$是参数的真实值。函数$\mu:R\times\Theta \rightarrow R$和$\sigma:R\times\Theta \rightarrow (0, \infty)$是已知的,并且使随机微分方程的解存在。如果我们观测到该随机微分方程解的一条轨迹的离散采样$\{X_n, n\in N\}$,用这些数据来估计参数$\theta$,得到参数的估计为$\hat{\theta}$。

假设$x_0,...,x_N$是$X(t)$在均匀离散时刻$t_i = \Delta t$的观测,其中$i = 0,1,...,N, \Delta t = T/N.$\\
令$p(t_k,x_k|t_{k-1},x_{k-1;\theta})$是从$(t_{k-1},x{k-1})$到$(t_k,x_k)$的传递概率密度,\\
假设初始状态的概率密度为$p_0(_0|\theta)$,似然函数为:$$f(\theta) = p_0(x_0|\theta)\prod_{k=1}^Np(t_k,x_k|t_{k-1},x_{k-1};\theta)$$
考虑SDE的Euler近似,有$$X(t_k) = x_{k-1} + \mu (t_{k-1},x_{k-1};\theta)\Delta t + \sigma(t_{k-1},x_{k-1};\theta)\sqrt{\Delta t}\eta_k$$
其中$\eta_k \sim N(0,1)$。因此
$$p(t_k,x_k|t_{k-1},x_{k-1};\theta) = \frac{1}{\sqrt{2\pi\sigma_k}}exp(-\frac{(x_k-\mu_k)^2}{2\sigma_k^2})$$
其中$\mu_k = x_{k-1} + \mu(t_{k-1},x_{k-1};\theta), \sigma_k = \sigma(t_{k-1},x_{k-1};\theta)\sqrt{\Delta t}$

当然,对于随机微分方程的参数估计还有高阶矩法、期望方差法等方法,这里就不一一介绍了。


\end{document}