\documentclass{ctexbeamer}



%usepackage{beamerthemeshadow} %该为一现成的模板,在 MiKTeXtexmftexlatexbeamerthemestheme下面有很多

\usetheme{Warsaw} %{Madrid}
\usepackage{amsmath}
\usepackage{graphicx}
\newtheorem{proposition}[theorem]{Proposition}
\renewcommand{\proofname}{标准布朗运动的数学性质}



\title{随机微分方程介绍}

\author{张恒}

\date{\today}



\begin{document} %申明文档的开始



    \begin{frame} 

        \titlepage   

    \end{frame}



    \begin{frame}

        \frametitle{大纲}

		\begin{itemize}
			\item 基本概念:\\
				随机过程、布朗运动、Ito积分、SDE...
			\item SDE的数值解
			\item SDE的参数估计
	
		\end{itemize}



    \end{frame}



    \begin{frame}

        \frametitle{随机过程}
		\small

		\textbf{例:} 我们在某一时段对某一地区成人的身高和体重(X, Y)进行随机抽样,可得出 
$$Z = (X,Y) \sim N(\mu_1,\mu_2,\rho,\sigma^2_1,\sigma^2_1)$$
$$Z(\omega) = (X(\omega),Y(\omega))$$

 这是某时段所考查随机变量的概率分布,如果\textit{每隔10年}在同一地区做同样的随机抽样,得: \\
$$Z(\omega,0)=(X(\omega,0),Y(\omega,0)) \sim N(\mu_{01},\mu_{02},\rho_0,\sigma^2_{01},\sigma^2_{02})$$

 第1个10年:$Z(\omega,1)=(X(\omega,1),Y(\omega,1)) \sim N(\mu_{11},\mu_{12},\rho_1,\sigma^2_{11},\sigma^2_{12})$

 第2个10年:$Z(\omega,2)=(X(\omega,2),Y(\omega,2)) \sim N(\mu_{21},\mu_{22},\rho_2,\sigma^2_{21},\sigma^2_{22})$

 ......

 第t个10年:$Z(\omega,t)=(X(\omega,t),Y(\omega,t)) \sim N(\mu_{t1},\mu_{t2},\rho_t,\sigma^2_{t1},\sigma^2_{t2})$ $t = 0,1,2...$

因此$\{Z(\omega,t)=(X(t),Y(t)): t \geq 0\}$表示的就是身高和体重这两个随机变量在不同时段的情况。


   \end{frame}


   \begin{frame}

        \frametitle{随机过程}

        \begin{definition}[随机过程]

            设 $(\Omega ,{\mathcal  {F}},P)$为一概率空间,另设集合T为一指标集合,如果对于所有$t\in T$,均有一随机变量 $X(t,\omega)$定义于概率空间$(\Omega ,{\mathcal  {F}},P)$,则集合$\{X(t,\omega)|t\in T\}$为一随机过程

        \begin{itemize}

         \item 对于固定的$\omega$, 比如$\overline{\omega}$, $\{X(t,\overline{\omega}), t \geq 0\}$被称为\textbf{路径(path)}或\textbf{轨迹(trajectory)}

         \item 对于固定的t,比如$\overline{t}$,集合$\{X(\overline{t},\omega), \omega \in \Omega\}$是时刻$\overline{t}$在该随机过程中的状态集,$X(\overline{t},\omega)$也就是时刻$\overline{t}$的随机变量


        \end{itemize}

        \end{definition}

    \end{frame}

	\begin{frame}

		\frametitle{布朗运动}	

		\begin{proof}
	
		设连续时间随机过程$W_t: 0 \leq t < T$ 是 $[0,T)$上的标准布朗运动,

			\begin{itemize}

			 \item $W_0 = 0$

			 \item \textbf{独立增量性:} 对于有限个时刻$0 \leq t_1 < t_2 < ... < t_n < T$,随机变量$$W_{t_2}-W_{t_1},W_{t_3}-W_{t_2},...,W_{t_n}-W_{t_{n-1}}$$是独立的

			 \item \textbf{正态性:} 对任意的$0 \leq s < t < T$, $W_t-W_s$服从均值为0,方差为t-s的正态分布

			\end{itemize}

		\end{proof}

	\end{frame}

	\begin{frame}

		\frametitle{随机微分方程}

		一般的随机微分方程$$\frac{\mathrm{d}X(t)}{\mathrm{d}t}=h[X(t),t]+g[X(t),t]R(t)$$
		形式上$$R(t)=\frac{\mathrm{d}B}{\mathrm{d}t}$$
		其中记号$\mathrm{d}W(t)=R(t)\mathrm{d}t$,$B(\omega,t)$是维纳过程 \\
		对上式求积分,得:$$X(t)-X(0)=\int_0^t h[s,X(s)]\mathrm{d}s + \int_0^t g[s,X(s)]\mathrm{d}B$$
		第一个积分是普通微积分的,第二个积分是维纳过程的随机函数的积分\\
		布朗运动:$\mathrm{d}X_t = \mu \mathrm{d}t + \sigma\mathrm{d}W_t$\\
		几何布朗运动:$\mathrm{d}X_t = \mu X_t\mathrm{d}t + \sigma X_t\mathrm{d}W_t$

	\end{frame}

	\begin{frame}

		\frametitle{随机积分的求解}
		\footnotesize  

		与普通微积分的牛顿莱布尼茨公式采用分区间近似求和相同,随机微积分中也是用分片的常数函数来近似$g[s,X(s)]$,即$$\int_0^tg[s,X(s)]\mathrm{d}B \approx \sum_{i=0}^{n-1}g_i[s,X(s)]\mathrm{d}B_i = \sum_{i=0}^{n-1}g_i[s,X(s)][B(t_{i+1})-B(t_i)], s \in [t_{i+1},t_i]$$
		其中s的取值有两种:

		\begin{itemize}

		\item 在区间左端点$t_i$取值$$g_i[s,X(s)]=G[t_i,X(t_i)]$$
			相应的随机积分$$\int_0^t g[s,X(s,\omega)]\mathrm{d}B(\omega,t)$$
			称为Ito积分

		\item 在区间端点取值求平均$$g_i[s,X(s)]=\frac{g[t_i,X(t_i)]+g[t_{i+1},X(t_{i+1})]}{2}$$
			相应的随机积分$$\int_0^t g[s,X(\omega,s)]∘\mathrm{d}B(\omega,t)$$
			称为Stratonovich积分

		\end{itemize}

	\end{frame}

	\begin{frame}

		\frametitle{随机积分的求解:例}
		\footnotesize
		设$g[t,B(t)]=B(t)$,则在Ito积分中:
		\begin{align}
		I_1 &= \int_0^t B(s)\mathrm{d}B \\ 
		    &\approx -\sum_{i=0}^{n-1}B_i(B_i-B_{i+1}) \\
			&=-[B_0^2-B_0B_1+B_0^2-B_0B_1+...+B_{n-1}^2-B_{n-1}B_n] \\
			&=-\frac{1}{2}[B_0^2+\sum_{i=1}^{n-1}(B_{i+1}-B_i)^2-B_n^2]=\frac{1}{2}(B_t^2-B_0^2)-\frac{1}{2}\sum_{i=1}^{n-1}\Delta B_i^2
		\end{align}
		
		在Stratonovich积分中:
		\begin{align}
		I_2 &= \int_0^t B(s)∘\mathrm{d}B \approx \sum_{i=0}{n-1}\frac{1}{2}[B(t_{i+1})+B(t_i)][B(t_{i+1})-B(t_i)]\\
			&= \frac{1}{2}\sum_{i=0}{n-1}[B^2(t_{i+1})-B^2(t_i)]\\
			&= \frac{1}{2}[B^2(t_1)-B^2(t_0)+B^2(t_2)-B^2(t_1)+...+B^2(t_n)-B^2(t_{n-1})]\\
			&= \frac{1}{2}[B^2(t_n)-B^2(t_0)] = \frac{1}{2}B^2(t)
		\end{align}

	\end{frame}

   \begin{frame}

		\frametitle{Ito公式}
		\footnotesize

		普通微积分中,若X连续可微,则$$X(t)-X(0)=\sum_{i=1}^n[X(t_i)-X(t_{i-1})]=\sum_{i=1}^nX'({\xi}_i)(t_i-t_{i-1})=\int_0^t X'(s)\mathrm{d}s$$

		将t替换成$B_t$,此时$\xi_i$是介于$B_{i-1}$和$B_i$之间的点,$\xi_i$在区间不同位置的取值会影响积分值,Ito将取${\xi_i}$为左端点进行求和,这就需要将泰勒公式多展开一阶
$X(B_{t_j})-X(B_{t_{j-1}}) = X'(B_{t_{j-1}})(B_{t_j}-B_{t_{j-1}})+\frac{1}{2}X''(\xi_j)(B_{t_j}-B_{t_{j-1}})^2$\\
		右端第一项求和,得$$\sum_{j=1}^nX'(B_{t_{j-1}})(B_{t_j}-B_{t_{j-1}}) \rightarrow \int_0^tX'(B_s)\mathrm{d}B_s$$\\
		右端第二项求和,$$\sum_{j=1}^n\frac{1}{2}X''(\xi_j)(B_{t_j}-B_{t_{j-1}})^2 \rightarrow \frac{1}{2}\int_0^tX''(B_s)\mathrm{d}s$$\\
		即 $X(B_t) = X(B_0) + \int_0^tX'(B_s)\mathrm{d}B_s + \frac{1}{2}\int_0^tX''(B_s)\mathrm{d}s$\\
		微分形式:$\mathrm{d}X(B_t) = X'(B_t)\mathrm{d}B_t + \frac{1}{2}F''(B_t)\mathrm{d}t$

    \end{frame}

	\begin{frame}
		\frametitle{Ito公式的应用}
		\footnotesize
		假设X(t)满足几何布朗运动的SDE:$$\mathrm{d}X_t = \mu X_t\mathrm{d}t + \sigma X_t\mathrm{d}W_t$$
		如何解出X(t)?\\
		设$Y(t) = lnX(t)$,则$\frac{\partial Y}{\partial X} = \frac{1}{X}$,$\frac{\partial^2Y}{\partial X^2} = -\frac{1}{X^2}$,由Ito公式得:
		\begin{align}
			\mathrm{d}Y &= \frac{\partial Y}{\partial X}\mathrm{d}X + \frac{1}{2}\frac{\partial^2Y}{\partial X^2}(\mathrm X_t)^2\\
					&= \frac{1}{X}(\mu X\mathrm{d}W) + \frac{1}{2}(-\frac{1}{X^2})\sigma^2X^2\mathrm{d}t\\
					&= \mu \mathrm{d}t + \sigma \mathrm{d}W - \frac{1}{2}\sigma^2\mathrm{d}t\\
					&= (\mu-\frac{1}{2}\sigma^2)\mathrm{d}t + \sigma\mathrm{d}W\\
		\end{align}


	\end{frame}

	\begin{frame}
	\frametitle{Ito公式的应用}
	\footnotesize
	则Y(t)是布朗运动,因此
	\begin{align}
Y(t) &= Y(t_0)+(\mu-\frac{1}{2}\sigma^2)(t-t_0) + \sigma(W(t)-W{t_0})\\
	X(t) &= exp(Y(t))\\
	   &= X(t_0)exp[(\mu-\frac{1}{2}\sigma^2)(t-t_0)+\sigma(W(t)-W(t_0))]
	\end{align}

	\end{frame}

	\begin{frame}

	\frametitle{SDE的数值解}
	\footnotesize
	
	并不是所有的SDE都能解出显式解,更多的SDE只能通过迭代式求数值解,求SDE数值解的过程也就是模拟出解的路径。
	\renewcommand{\proofname}{Euler格式}
	\begin{proof}
	$$X_{i+1} = X_i + \mu (t_i,X_i)(t_{i+1-t_i}) + \sigma(t_i,X_i)(W_{i+1}-W_i)$$
	\end{proof}

	\renewcommand{\proofname}{Milstein格式}
	\begin{proof}
	\begin{align}
	X_{i+1} &= X_i + \mu (t_i,X_i)(t_{i+1}-t_i)+\sigma (t_i,X_i)(W_{i+1}-W_i)\\
			&\frac{1}{2}\sigma(t_i,X_i)\sigma_y(t_i,X_i)\{(W_{i+1}-W_i)^2-(t_{i+1}-t_i)\}
	\end{align}

	\end{proof}

	其中$$W_{i+1}-W_i = \sqrt{t_{i+1}-t_i}Z_{i+1}, i=0,1,...,n-1$$
	而$Z_1,...,Z_n$是相互独立的标准正态随机变量

	\end{frame}

	\begin{frame}

	\frametitle{SDE的参数估计}
	\begin{itemize}
	\item 极大似然估计
	\end{itemize}
	以布朗运动$\mathrm{d}X_t = \mu(X_t;\theta)\mathrm{d}t + \sigma(X_t;\theta)\mathrm{d}W_t$为例
	假设$x_0,...,x_N$是$X(t)$在均匀离散时刻$t_i = \Delta t$的观测,其中$i = 0,1,...,N, \Delta t = T/N.$\\
	令$p(t_k,x_k|t_{k-1},x_{k-1;\theta})$是从$(t_{k-1},x{k-1})$到$(t_k,x_k)$的传递概率密度,\\
	假设初始状态的概率密度为$p_0(_0|\theta)$,似然函数为:$$f(\theta) = p_0(x_0|\theta)\prod_{k=1}^Np(t_k,x_k|t_{k-1},x_{k-1};\theta)$$
	考虑SDE的Euler近似,有$$X(t_k) = x_{k-1} + \mu (t_{k-1},x_{k-1};\theta)\Delta t + \sigma(t_{k-1},x_{k-1};\theta)\sqrt{\Delta t}\eta_k$$
	其中$\eta_k \sim N(0,1)$。因此
	$$p(t_k,x_k|t_{k-1},x_{k-1};\theta) = \frac{1}{\sqrt{2\pi\sigma_k}}exp(-\frac{(x_k-\mu_k)^2}{2\sigma_k^2})$$
	其中$\mu_k = x_{k-1} + \mu(t_{k-1},x_{k-1};\theta), \sigma_k = \sigma(t_{k-1},x_{k-1};\theta)\sqrt{\Delta t}$

	\end{frame}

	\begin{frame}

	\frametitle{SDE的参数估计}
	\begin{itemize}
	\item 实例:种群动力学
	\end{itemize}

	考虑美洲鹤1939-1985年的种群数据,假设t时刻种群大小X(t)满足SDE:$$\mathrm{d}X(t) = \theta_1X(t)\mathrm{d}t + \theta_2\sqrt{X(t)}\mathrm{d}W(t), X(0) = 18$$

	\begin{center}
	\includegraphics[width = .6\textwidth]{goose.png}
	\end{center}

	\end{frame}


\end{document}