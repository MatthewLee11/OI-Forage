\documentclass[10pt,aspectratio=169,mathserif]{beamer}		
%设置为 Beamer 文档类型,设置字体为 10pt,长宽比为16:9,数学字体为 serif 风格

%%%%-----导入宏包-----%%%%
\usepackage{ctex}			 %导入 ctex 宏包,添加中文支持
\usepackage{amsmath,amsfonts,amssymb,bm}   %导入数学公式所需宏包
\usepackage{color}			 %字体颜色支持
\usepackage{graphicx,hyperref,url}	


\beamertemplateballitem		%设置 Beamer 主题

%%%%------------------------%%%%%
\catcode`\。=\active         %或者=13
%将正文中的“。”号转换为“.”。
%%%%%%%%%%%%%%%%%%%%%

%%%%----首页信息设置----%%%%
\title{levy 过程}

\begin{document}

\begin{frame}
\titlepage
\end{frame}				%生成标题页

\begin{frame}
\tableofcontents
\end{frame}				%生成提纲页

\section{定义}
\begin{frame}
  \frametitle{定义}
  一个随机过程 ${X=\{X_{t}:t\geq 0\}}$如果符合以下条件: 
  \begin{enumerate}[(1)]
    \item $X_{0}=0$.
    \item 独立增量:对于任何$0\leq t_{1} < t_{2} < \dots < t_{n} < \infty ,X_{t_2}-X_{t_1},X_{t_3}-X_{t_2},\dots,X_{t_n}-X_{t_{n-1}}$相互独立.
    \item 稳定增量:对任何s<t, $X_{t}-X_{s}$ 与 $X_{t-s}$有相同分布 .  
     \item X几乎确定是右连左极.
     \end{enumerate}
     我们把X称作levy过程.
\end{frame}

\section{无限可分分布(infinitely divisible distribution)}
\begin{frame}
  \frametitle{无限可分分布}
  \begin{itemize}
  \item 定义:对于一个随机变量U,如果存在一系列的独立同分布的随机变量$U_{1,n},\dots,U_{n,n}$,使得U$\overset{\text{d}}{=}U_{1,n},\dots,U_{n,n}$,那么随机变量U拥有无限可分分布,$\overset{\text{d}}{=}$意味着同分布.
  \item 如果U是一个随机变量,那么她的特征函数h:R $\to$ C为:h($\theta$)=E[$e^{i\theta U}$],$\theta \in R$.
  \item U是一个无限可分随机变量,h是特征函数,那么:
  	\begin{enumerate}[(1)]
	\item h是连续并且非0的,h(0)=1.
	\item 存在一个独特的连续函数,对于所有$\theta \in R$满足$e^{f(\theta)}=h(\theta)$,f是U的特征指数
	\end{enumerate}
  \end{itemize}
\end{frame}

\begin{frame}
	 定理:如果X是一个levy过程,对于任何$t \geq 0$,随机变量 $X_{t_1} $ 拥有无限可分分布.
	 证明:如果X是kevy过程,并且$t \geq 0$.那么对于n=1,2,\dots,\[ X_{t}=X_{t/n}+(X_{2t/n}-X_{t/n})+\dots+(X_{t}-X{(n-1)t/n})\]右边的每项都是独立同分布,并且有独立稳定增量属性,因此,${X_{t}}$拥有无限可分分布.
	 接下来,我们看下一维levy过程的特征函数,对于levy过程X,$X_{t}$的特征指数为$\psi_{t}$,\[e^{\psi_{t}(\theta)}=E[e^{i \theta X_{t}}] ,\theta \in R \]通过上面的公式,我们可以得到\[ m\psi_{1}(\theta)=\psi_{m}(\theta)=n\psi_{m/n}(\theta)\] 也就是说对于$t \geq 0$,\[\psi_{t}(\theta)=t\psi_{1}(\theta)\]因此,我们可以得到\[E[e^{i\theta X_{t}}]=e^{t\psi(\theta)}, \theta \in R\]其中,$\psi$:=$\psi_{1}$是$X_{1}$的特征指数.
	 
\end{frame}
\section{莱维-辛钦定理}
\begin{frame}
  \frametitle{莱维-辛钦定理}
   莱维-辛钦定理:X为levy过程,特征指数为$\psi$.存在唯一的$a \in R,\sigma \geq 0$以及测度$\prod$,满足$\int_R 1 \bigwedge x \prod \,dx$  \[\psi(\theta)=ia\theta-\frac{1}{2}\sigma_{2}\theta_{2}+(e^{i\theta x}-1-i\theta xI_{[-1,1]}(x))\prod(dx)\]对于上述三个参数($a,\sigma,\prod$),a包含了任何确定的漂移项,$\sigma$是高斯系数,描述了布朗运动的波动性,而$\prod$描述了X跳跃的大小和强度
\end{frame}

\section{levy过程的特例}
\begin{frame}
\frametitle{possion 过程}
  possion过程的概率密度函数为$\mu_{\lambda({k})}=e^{-\lambda}\lambda^{k}/k!$,因此possion分布的特征函数为\[\sum_{k=1}^{n}e^{i\theta k}\mu_{\lambda}({k})=e^{\lambda(e^{i\lambda}-1)}=[e^{\frac{\lambda}{n}}(e^{i\theta}-1)]^{n} \]因此对于一个possion过程{$N_{t}:t \geq 0$}如果参数为$\lambda t$,则是levy过程,特征函数为$E(e^{i\theta N_{t}})=e^{\lambda t(e^{i\theta}-1)}$,特征指数为$\psi(\theta)=\lambda(e^{i\theta}-1)$
 \end{frame}

\begin{frame}
复合possion过程\newline\newline
  N为possion过程,对于$\theta \in R $,特征函数为\[E(e^{i\theta\sum_{i=1}^{N}\xi_{i}})=\sum_{n\geq0}E(e^{i\theta\sum_{i=1}^{N}\xi_{i}})e^{-\lambda}\frac{\lambda^{n}}{n!}
  =\sum_{n\geq0}(\int_R e^{i\theta x}F(dx)\,)^{n}e^{-\lambda}=e^{\lambda\int_R e^{i\theta x}F(dx)\,} \]其中三个参数分别为$a=\lambda\int_{0<|x|<1}xF(dx) \,$,$\sigma=0$,$\prod(dx)=\lambda F(dx)$.

\end{frame}

\begin{frame}
布朗运动\newline\newline
布朗运动的概率密度函数为:\[\mu_{s,\gamma}(dx):=\frac{1}{\sqrt{2\pi s^{2}}}e^{-(x-\gamma)^{2}/2s^{2}}dx ,x \in R\]我们可以得到他的特征函数和特征指数:\[E[i\theta U]=e^{\frac{1}{2}s^{2}\theta^{2}+i\theta\gamma}\]三个参数为$a=-\gamma,\sigma=s,\prod=0$

\end{frame}

\section{莱维-伊藤分解}
\begin{frame}
 \frametitle{莱维-伊藤分解}


根据莱维-辛钦定理,我们可以得到如下公式\[\Psi(\theta)=\{ia\theta+\frac{1}{2}\sigma^{2}\theta^{2}\}+\{\prod(R\backslash(-1,1))\int_{|x|\geq1}(1-e^{i\theta x})\frac{\prod{dx}}{\prod(R\backslash(-1,1)} \,\}\]\newline\[+\{\int_{0<|x|<1}(1-e^{i\theta x}+i\theta x)\prod(dx) \,\} \]
第一项是线性布朗运动,第二项是复合泊松过程,第三项是平方可积鞅.因此levy过程可由这三个过程组合而成.

\end{frame}

\end{document}