%Chaoming Gu, 2018
\documentclass[UTF8]{beamer}
\usepackage{ctex}
\usetheme{Madrid}
\renewcommand\refname{参考文献}



\begin{document}
	
\begin{frame}
\title{觅食算法}
\titlepage
\end{frame}

\begin{frame}{目录}
\tableofcontents
\end{frame}


\section{觅食算法实例}

\begin{frame}
\frametitle{觅食算法实例}
\framesubtitle{概览}
	\begin{itemize}
	\item 资源分配和调度
	\item 信号处理
	\item 预测控制
	\item 图形图像
	\end{itemize}
\end{frame}

\begin{frame}
\frametitle{觅食算法实例}
\framesubtitle{资源分配和调度}
\begin{itemize}
	\item 对计算机中的多区域温度调度平台(\textit{Multizone Temperature Experimentation Platform})进行资源分配,以低温区域为“食物”,细菌的觅食过程就是将数据的获取与计算工作分配至低温区域以获得更高的能效
	\item 车间作业中利用觅食算法进行空闲时间片段的调度,将未完成的工序合理地分配到不同的空闲时间中。在工序顺序和机器占用的约束下,有效降低调度的规模和复杂性
\end{itemize}
\end{frame}

\begin{frame}
\frametitle{觅食算法实例}
\framesubtitle{信号处理}
独立组分分析(ICA)用来对统计信号进行处理,找到非高斯分布数据的线性表示。利用觅食算法与ICA的结合,可以提高信号处理的收敛速度,降低均方误差
\end{frame}

\begin{frame}
\frametitle{觅食算法实例}
\framesubtitle{预测控制}
\begin{itemize}
	\item 利用觅食算法对股票市场指数趋势进行预测
	\item 利用觅食算法对控制系统进行局部优化
\end{itemize}
\end{frame}

\begin{frame}
\frametitle{觅食算法实例}
\framesubtitle{图形图像}
利用细菌觅食算法,在图像识别中,提高识别的速度与准确率。通常会与基因算法等其他算法进行结合,在面部识别等领域提高特征点的搜索速度。
\end{frame}


\begin{frame}
\frametitle{参考文献}
\begin{thebibliography}{widestlabel}
	content...
\end{thebibliography}
\end{frame}
\end{document}